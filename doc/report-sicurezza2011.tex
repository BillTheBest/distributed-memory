\documentclass[a4paper]{article}
% Codifica italiana e sillabazione
\usepackage[italian]{babel}
\usepackage[utf8]{inputenc}
% other
\usepackage{url}
\usepackage[pdftex,colorlinks=true,linkcolor=blue]{hyperref}
\usepackage{ban-gny}
\usepackage{amsthm}
% metadata
\newcommand{\Title}{Progetto di Sicurezza nei Sistemi Informatici \\
  {\large \textbf{Anno Accademico 2010-2011}}}
\newcommand{\Author}{Francesco Disperati, Alessio Celli}
\hypersetup{
  pdftitle={\Title},
  pdfauthor={\Author},
}
\title{\Title}
\author{\Author}
\date{\today}


% Document start
\begin{document}
\maketitle

\section{Analisi dei requisiti}
Viene richiesta la progettazione e l'implementazione di un protocollo per la comunicazione cifrata tra client-server. Requisiti:
\begin{itemize}
\item il server possiede una coppia di chiavi pubblica/privata
\item la chiave pubblica è nota ai client (scambio out-of-band)
\item i client condividono una passphrase segreta con il server
\end{itemize}
al termine del protocollo, client e server devono disporre di una chiave di sessione simmetrica con cui comunicare.


\section{Analisi del protocollo}

\begin{tabular}{ p{4cm} p{4cm} p{4cm} }
  {\bf Legenda}\\
  $A$     client process &
  $K_S$   session key &
  $K_{B}$ server public key \\

  $B$ server process &
  $K$ shared secret &
  $K_{B}^{-1}$ server private key \\

  $N_A$ nonce 
\end{tabular}


\paragraph{Protocollo reale}
\begin{list}{}{}
\item M1. $ A \rightarrow B: E_{K_{B}}(N_A, K_S, K) $
\item M2. $ B \rightarrow A: E_{K_S}(N_A, K_S) $
\end{list}
L'handshake consiste di due semplici passaggi. $A$ genera un nonce e la chiave di sessione, e li spedisce assieme al segreto $K$ a $B$, criptandoli con la sua chiave pubblica. $B$ decripta il messaggio, verifica la correttezza di $K$, e invia ad $A$ il nonce e la chiave di sessione criptati con la chiave di sessione stessa. $A$ verifica la correttezza della risposta e a questo punto entrambi i processi possono utilizzare la chiave di sessione.


\paragraph{Protocollo ideale}
\begin{list}{}{}
\item M1. $ A \rightarrow B: \encrypt{N_A, \sharedkey{A}{B}{K_S}, \secret{A}{B}{K}}{K_B} $
\item M2. $ B \rightarrow A: \encrypt{N_A, \sharedkey{A}{B}{K_S} }{K_S} $
\end{list}

Vogliamo dimostrare la {\bf key authentication}
\begin{align}
  \believes{B}{ \sharedkey{A}{B}{K_S} }& \qquad
  \believes{A}{ \sharedkey{A}{B}{K_S} } \tag{T1}
\end{align}

e la {\bf key confirmation}
\begin{align}
  \believes{B}{ \believes{A}{ \sharedkey{A}{B}{K_S} } }& \qquad
  \believes{A}{ \believes{B}{ \sharedkey{A}{B}{K_S} } } \tag{T2}
\end{align}

\begin{tabular}{ p{4cm} p{4cm} p{4cm} }

  \parbox[t][][t]{4cm} {
    {\bf Ipotesi secrecy}
    \begin{align}
      \believes{A&}{ \publickey{K_B}{B} }\\
      \believes{A&}{ \sharedkey{A}{B}{K_S} }\\
      \believes{A&}{ \secret{A}{B}{K} }\\
      \believes{B&}{ \secret{A}{B}{K} }
    \end{align}
  }&

  \parbox[t][][t]{4cm} {
    {\bf Ipotesi freshness}
    \begin{align}
      \believes{A&}{ \fresh{N_A} }\\
      \believes{B&}{ \fresh{N_A} }
    \end{align}
  }&

  \parbox[t][][t]{4cm} {
    {\bf Ipotesi mutual trust}
    \begin{align}
      \believes{B&}{ \jurisdiction{A}{ \sharedkey{A}{B}{K_S} } }
    \end{align}
  }
\end{tabular} \\

\begin{proof}
Per l'ipotesi 2 abbiamo subito
\[
\believes{A}{ \sharedkey{A}{B}{K_S} }  \tag{ {\bf key authentication} for $A$ }
\]

Dopo M1 $\sees{B}{K}$, per l'ipotesi 4 e la {\em message meaning rule} otteniamo $ \believes{B}{ \said{A}{( \sharedkey{A}{B}{K_S}, N_A )} } $.

Inoltre per l'ipotesi 6 e la {\em nonce verification rule}
\[
\believes{B}{ \believes{A}{ \sharedkey{A}{B}{K_S} } }   \tag{ {\bf key confirmation} for $B$}
\]

Per l'ipotesi 7 e la {\em jurisdiction rule} otteniamo anche
\[
\believes{B}{ \sharedkey{A}{B}{K_S} } \tag{ {\bf key authentication} for $B$ }
\]

Dopo M2 $\sees{A}{ \decrypt{N_A,K_S}{K_B} }$, ovvero per l'ipotesi 1 $\sees{A}{ (N_A,K_S) }$. Applicando la {\em message meaning rule}
\[
\believes{A}{ \believes{B}{ \sharedkey{A}{B}{K_S} } } \tag{ {\bf key confirmation} for $A$ }
\]
concludiamo la dimostrazione.
\end{proof}


\section{Implementazione}
Per la realizzazione del protocollo sono state utilizzate le librerie OpenSSL. Lo scambio di messaggi a chiave pubblica-privata è basato su RSA, mentre per la comunicazione simmetrica viene utilizzato AES-256 in modalità ECB. Per una descrizione dettagliata dell'implementazione fare riferimento alla documentazione ({\tt doc/doxygen/html/}).


\end{document}
